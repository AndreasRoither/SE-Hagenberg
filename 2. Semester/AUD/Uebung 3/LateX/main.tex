%für Sprache, A4 Blatt, float, Grafiken, UTF Codierung, PDF, Color, Seitenabstand, Listings
\documentclass[a4papr,12pt]{article}
\usepackage[utf8]{inputenc}
\usepackage[ngerman]{babel}
\usepackage{graphicx}
\usepackage{float}
\usepackage{textcomp}
\usepackage{pdfpages}
\usepackage{tikz}
\usepackage{hyperref}
\usepackage{geometry}
\usepackage{listings}
\usepackage{color}
\usepackage{grffile}
\usepackage{caption}

%Mathematics
\usepackage{amstext}
\usepackage{amssymb}
\usepackage{amsmath}
\usepackage{amsfonts}
\usepackage{mathrsfs}
\usepackage{mathtools}

%include this before fancy or page style gets messed up bc of geometry
\include{settings}

%Für Kopfzeile den Style
\usepackage{fancyhdr}
\pagestyle{fancy}
\lhead{AUD 2\textbackslash PRO 2 - Übung 3}
\rhead{Andreas Roither, \today{}}
\newcommand{\Cross}{\mathbin{\tikz [x=1.4ex,y=1.4ex,line width=.2ex] \draw (0,0) -- (1,1) (0,1) -- (1,0);}}%

\begin{document}

%ANGABE     
\thispagestyle{plain}
\includepdf[pages={1},pagecommand={     
\begin{tikzpicture}[remember picture, overlay]\node at (15.8, -0.85) {6 h};\end{tikzpicture}
\begin{tikzpicture}[remember picture, overlay]\node at (7.6, -0.85) {Andreas Roither};\end{tikzpicture}
\begin{Huge}
\begin{tikzpicture}[remember picture, overlay]\node at (-1.3, -1.9) {X};\end{tikzpicture}
\end{Huge}
}]{Angabe/UE3.pdf}
\thispagestyle{plain}
\includepdf[pages=2-,pagecommand={}]{Angabe/UE3.pdf}

\section*{Übung 3}
\subsection*{Aufgabe 1}
\subsubsection*{Lösungsidee}
Bei der Dekompression und der Kompression wird Zeichen für Zeichen überprüft ob es eine Nummer oder ein anderer Charakter ist. Falls eine Nummer bei der Dekompression  vorkommt, wird das vorherige Zeichen dementsprechend oft eingefügt. Bei der Kompression wird die Anzahl des nacheinander auftretenden Buchstabens gezählt und mit dem Buchstaben in die txt Datei geschrieben. Bei der Kompression und anschließender Dekompression kann eine Zahl bei Sonderkombinationen zu einem falschen Ergebnis führen. Bsp: \grqq{}aaaa1\grqq{} würde bei einer Umwandlung \grqq{}a41\grqq{} werden. Bei anschließender Dekompression werden jedoch daraus 41 a und nicht wie vorher 4 a Zeichen.
\newline

\lstinputlisting[language=Pascal] {../rle.pas}
\begin{figure}[H]
	\centering
	\includegraphics[scale=0.75]{./pictures/1.png}
	\caption{RLE Optionen}
	\label{fig: Options}
\end{figure}
Zu sehen sind die zwei verschiedenen Optionen und die Eingabeaufforderung falls nichts angegeben wurde.\newline

\lstinputlisting[language=] {../d.txt}
\lstinputlisting[language=] {../c.txt}

In d.txt ist der dekomprimierte Text, in c.txt der komprimierte.

\newpage
\subsection*{Aufgabe 2}
\subsubsection*{Lösungsidee}
Zuerst werden die Wörter die ersetzt werden sollen eingelesen und in einer Hash Tabelle gespeichert. Jedes Zeichen in der Ostern.txt wird eingelesen und vorkommende Wörter werden überprüft. Der berechnete Hashwert eines Wortes wird mit einem Element in der Hash Tabelle an der Position des Hashwertes verglichen. Falls ein Element mit dem Hashcode existiert und genau dasselbe Wort ist (Vergleich, falls ein Wort mit dem selben Hashwert existiert jedoch nicht Zeichen für Zeichen gleich ist), wird das ersetzende Wort in die neue txt Datei eingefügt, andernfalls wird das ursprüngliche Wort eingefügt.   
\newline

\lstinputlisting[language=Pascal] {../storygen.pas}
\begin{figure}[H]
	\centering
	\includegraphics[scale=0.75]{./pictures/2.png}
	\caption{Storygen}
	\label{fig: Matching}
\end{figure}
\newpage
\lstinputlisting[] {../replace.txt}
\raggedright
In der replace.txt werden die Wörter die ersetzt werden sollen und die Wörter durch die sie ersetzt werden sollen gespeichert, getrennt durch ein beliebiges Trennzeichen (außer Buchstaben).
\newline

\lstinputlisting[linerange={1-8}] {../Ostern.txt}
\lstinputlisting[linerange={1-8}] {../new.txt}
\raggedright
Zu sehen sind die ersten 8 Zeilen der Ostern.txt und der new.txt. Wörter, die in der replace.txt vorgegeben sind, wie Osterhasen oder Hühnchen wurden durch die in der replace.txt vorhandenen Wörtern ersetzt.

\end{document}





