%für Sprache, A4 Blatt, float, Grafiken, UTF Codierung, PDF, Color, Seitenabstand, Listings
\documentclass[a4papr,12pt]{article}
\usepackage[utf8]{inputenc}
\usepackage[ngerman]{babel}
\usepackage{graphicx}
\usepackage{float}
\usepackage{textcomp}
\usepackage{pdfpages}
\usepackage{tikz}
\usepackage{hyperref}
\usepackage{geometry}
\usepackage{listings}
\usepackage{color}
\usepackage{grffile}
\usepackage{caption}

%Mathematics
\usepackage{amstext}
\usepackage{amssymb}
\usepackage{amsmath}
\usepackage{amsfonts}
\usepackage{mathrsfs}
\usepackage{mathtools}

%include this before fancy or page style gets messed up bc of geometry
\include{settings}

%Für Kopfzeile den Style
\usepackage{fancyhdr}
\pagestyle{fancy}
\lhead{AUD 2\textbackslash PRO 2 - Übung 2}
\rhead{Andreas Roither, \today{}}
\newcommand{\Cross}{\mathbin{\tikz [x=1.4ex,y=1.4ex,line width=.2ex] \draw (0,0) -- (1,1) (0,1) -- (1,0);}}%

\begin{document}

%ANGABE     
\thispagestyle{plain}
\includepdf[pages={1},pagecommand={     
\begin{tikzpicture}[remember picture, overlay]\node at (15.8, -0.85) {6 h};\end{tikzpicture}
\begin{tikzpicture}[remember picture, overlay]\node at (7.6, -0.85) {Andreas Roither};\end{tikzpicture}
\begin{Huge}
\begin{tikzpicture}[remember picture, overlay]\node at (-1.3, -1.9) {X};\end{tikzpicture}
\end{Huge}
}]{Angabe/UE2.pdf}
\thispagestyle{plain}
\includepdf[pages=2-,pagecommand={}]{Angabe/UE2.pdf}

\section*{Übung 2}
\subsection*{Aufgabe 1}
\subsubsection*{Lösungsidee}
Bei MinM wird in eine Liste eingefügt falls ein Zeichen darin noch nicht enthalten ist. Am Schluss wird gezählt wie viele Zeichen in der Liste enthalten sind. Diese Anzahl gibt MinM zurück. Bei MaxMStringLen wird solange in eine Liste eingefügt bis m erreicht wird. Falls die Anzahl unterschiedlicher Zeichen in der Liste überschritten wurde, löscht RemoveFirst das erste Zeichen in der Liste. Falls die aktuelle Länge größer ist als die vorher gespeicherte Länge wird die aktuelle Länge gespeichert.
\newline

\lstinputlisting[language=Pascal] {../Kette.pas}
\begin{figure}[H]
	\centering
	\includegraphics[scale=0.75]{./pictures/1a.png}
	\caption{Ausgabe 1a}
	\label{fig: Kette1a}
\end{figure}
Ausgegeben wird die Anzahl der unterschiedlichen Zeichen der oben angegebenen Zeichenketten.

\begin{figure}[H]
	\centering
	\includegraphics[scale=0.75]{./pictures/1b.png}
	\caption{Ausgabe 1b}
	\label{fig: Kette1b}
\end{figure}
Hier wird der längste Substring mit maximalen m eines Strings ausgegeben.

\newpage
\subsection*{Aufgabe 2}
\subsubsection*{Lösungsidee}
Das Matching funktioniert ähnlich wie die BruteForce Methode. Es wird von links nach rechts durch den string gegangen. Dabei wird bei jedem Aufruf ein Teil des string nicht mehr mit übergeben, für die Zeichen \grqq{}*\grqq{} und \grqq{}?\grqq{} werden dabei spezielle Aktionen durchgeführt. Falls die Zeichenfolge nicht mehr übereinstimmt wird False zurückgegeben, andernfalls wird True zurückgegeben.
\newline

\lstinputlisting[language=Pascal] {../wildcard.pas}
\begin{figure}[H]
	\centering
	\includegraphics[scale=0.75]{./pictures/2arecursive.png}
	\caption{Ausgabe Matching}
	\label{fig: Matching}
\end{figure}
Ausgabe für die jeweiligen Zeichenketten.

\end{document}





