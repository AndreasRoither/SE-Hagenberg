%für Sprache, A4 Blatt, float, Grafiken, UTF Codierung, PDF, Color, Seitenabstand, Listings
\documentclass[a4papr,12pt]{article}
\usepackage[utf8]{inputenc}
\usepackage[ngerman]{babel}
\usepackage{graphicx}
\usepackage{float}
\usepackage{textcomp}
\usepackage{pdfpages}
\usepackage{tikz}
\usepackage{hyperref}
\usepackage{geometry}
\usepackage{listings}
\usepackage{color}
\usepackage{booktabs}

%Mathematics
\usepackage{amstext}
\usepackage{amssymb}
\usepackage{amsmath}
\usepackage{amsfonts}
\usepackage{mathrsfs}
\usepackage{mathtools}

%include this before fancy or page style gets messed up bc of geometry
\include{settings}

%Für Kopfzeile den Style
\usepackage{fancyhdr}
\pagestyle{fancy}
\lhead{1. Semester - Software Engineering}
\rhead{Andreas Roither, \today{}}

\begin{document}

%ANGABE 
%\thispagestyle{plain}
%\includepdf[pages={1},pagecommand={     
%\begin{tikzpicture}[remember picture, overlay]\node at (15.8, -1.35) {4 h};\end{tikzpicture}
%\begin{tikzpicture}[remember picture, overlay]\node at (7.6, -1.35) {Andreas Roither};\end{tikzpicture}
%\begin{Huge}
%\begin{tikzpicture}[remember picture, overlay]\node at (-1, -1.9) {X};\end{tikzpicture}
%\end{Huge}
%}]{Angabe/Uebung03.pdf}
%\includepdf[pages=2-,pagecommand={}]{testpdf}

% Makros fuer wahr/falsch in der Wahrheitstabelle:
\newcommand{\wahr}{\text{w}}
\newcommand{\falsch}{\text{f}}
% kann leicht geändert werden auf andere Darstellung, etwa mit 0 und 1:
%\newcommand{\wahr}{1}
%\newcommand{\falsch}{0}

\title{1. Semester - Software Engineering}
\author{Andreas Roither}
\date{bla{}, Hagenberg}
\maketitle
\newpage

\tableofcontents
\newpage


%LGI Übung%

\includepdf[offset=-10 -45,pages=1,pagecommand={\section{LGI Übung} \subsection{Übung 1}}]{Angabe/LGI_Uebung01.pdf}
\includepdf[pages=2-,offset=-10 -45,pagecommand={}]{Angabe/LGI_Uebung01.pdf}
\begin{enumerate}
\item
\begin{enumerate}
\item $\conj{\disj{((A\lor\conj{(B\land C)}}\land\nega{(\neg \impl{(A \Rightarrow B)}))})}$
\item $\impl{(\conj{(A\land B)}\Rightarrow\disj{(A\lor\nega{(\neg B)}}))}$
\item $\disj{\conj{(\impl{(A\Rightarrow B)}\land C)} \conj{\nega{(\neg A)}\land\nega{(\neg C)}}}$
\end{enumerate}
\item
\begin{enumerate}
\item Aussage A endet mit Implikation (Junktor) -> nicht wohlgeformt
\item eine Klammer zu viel
\item es fehlen umhüllende Klammern
\end{enumerate}
\item 
\begin{enumerate}
\item Wahrheitstabelle

\begin{tabular}{|c|c|c|c|c|}
\hline
$A$ & $B$ & $C$ & $B \land C$ & $(A \lor (B \land C)))$\\ \hline
\falsch & \falsch 	& \falsch 	& \falsch	& \falsch	\\ \hline 
\falsch & \falsch 	& \wahr 	& \falsch 	& \falsch	\\ \hline 
\falsch & \wahr 	& \falsch 	& \falsch 	& \falsch	\\ \hline 
\falsch & \wahr 	& \wahr 	& \wahr 	& \wahr		\\ \hline 
\wahr 	& \falsch 	& \falsch 	& \falsch 	& \wahr		\\ \hline 
\wahr 	& \falsch 	& \wahr 	& \falsch 	& \wahr		\\ \hline 
\wahr 	& \wahr 	& \falsch 	& \falsch 	& \wahr		\\ \hline 
\wahr 	& \wahr 	& \wahr 	& \wahr 	& \wahr		\\ \hline 
\end{tabular}

\item Wahrheitstabelle \\
\begin{tabular}{|c|c|c|c|c|}
\hline
$A$ & $B$ & $A \land B$ & $A \lor B$ & $((A \land B) \Rightarrow (A \lor B))$\\ \hline
\falsch & \falsch 	& \falsch 	& \falsch	& \wahr		\\ \hline 
\falsch & \falsch 	& \falsch 	& \wahr 	& \wahr		\\ \hline 
\wahr 	& \wahr 	& \falsch 	& \wahr 	& \wahr		\\ \hline 
\wahr	& \wahr 	& \wahr 	& \wahr 	& \wahr		\\ \hline 
\end{tabular}
\\\\
Tautologie.. 	Wenn alle in einer Spalte Richtig sind\\
Widerspruch.. 	Wenn alle in einer Spalte Falsch sind

\item Wahrheitstabelle \\
\begin{tabular}{|c|c|c|c|c|c|}
\hline
$A$ & $B$ & $C$ & $(A \Rightarrow B)$ & $((A \Rightarrow B) \land C)$ & $((A \Rightarrow B)\land C) \lor(\neg A))$\\ \hline
\falsch & \falsch 	& \falsch 	& \wahr		& \falsch	& \wahr		\\ \hline 
\falsch & \falsch 	& \wahr 	& \wahr 	& \wahr		& \wahr		\\ \hline 
\falsch & \wahr 	& \falsch 	& \wahr 	& \falsch	& \wahr		\\ \hline 
\falsch & \wahr 	& \wahr 	& \wahr 	& \wahr		& \wahr		\\ \hline 
\wahr 	& \falsch 	& \falsch 	& \falsch 	& \falsch	& \falsch	\\ \hline 
\wahr 	& \falsch 	& \wahr 	& \falsch 	& \falsch	& \falsch	\\ \hline 
\wahr 	& \wahr 	& \falsch 	& \wahr 	& \falsch	& \falsch	\\ \hline 
\wahr 	& \wahr 	& \wahr 	& \wahr 	& \wahr		& \wahr		\\ \hline 
\end{tabular}
\end{enumerate}

\newpage
\item
\begin{enumerate}
\item Wahrheitstabelle\\
\begin{tabular}{|c|c|c|c|c|c|c|}
\hline
$A$ & $B$ & $C$ & $(A \lor(\neg B \land A))$ & $(C \lor (B \lor A))$ & $(1 \land 2)$ & $A$\\ \hline
\falsch & \falsch 	& \falsch 	& \falsch	& \falsch	& \falsch	& \falsch	\\ \hline 
\falsch & \falsch 	& \wahr 	& \falsch 	& \wahr		& \falsch	& \falsch	\\ \hline 
\falsch & \wahr 	& \falsch 	& \falsch 	& \wahr		& \falsch	& \falsch	\\ \hline 
\falsch & \wahr 	& \wahr 	& \falsch 	& \wahr		& \falsch	& \falsch	\\ \hline 
\wahr 	& \falsch 	& \falsch 	& \wahr 	& \wahr		& \wahr		& \wahr		\\ \hline 
\wahr 	& \falsch 	& \wahr 	& \wahr 	& \wahr		& \wahr		& \wahr		\\ \hline 
\wahr 	& \wahr 	& \falsch 	& \wahr 	& \wahr		& \wahr		& \wahr		\\ \hline 
\wahr 	& \wahr 	& \wahr 	& \wahr 	& \wahr		& \wahr		& \wahr		\\ \hline 
\end{tabular}
\\\\
Wenn zwei Spalten die selben Werte haben(wahr,falsch) dann sind diese Ident $\equiv$
\item Wahrheitstabelle\\
\begin{tabular}{|c|c|c|c|c|c|c|c|}
\hline
$x$ & $y$ & $z$ & $\neg(x\land \neg y)$ & $(y\land (x \lor z))$ & $(1 \lor 2)$ & $\neg x\lor y$ & $x\Rightarrow y$\\ \hline
\falsch & \falsch 	& \falsch 	& \wahr		& \falsch	& \wahr		& \wahr		& \wahr		\\ \hline 
\falsch & \falsch 	& \wahr 	& \wahr 	& \falsch	& \wahr		& \wahr		& \wahr		\\ \hline 
\falsch & \wahr 	& \falsch 	& \wahr 	& \falsch	& \wahr		& \wahr		& \wahr		\\ \hline 
\falsch & \wahr 	& \wahr 	& \wahr 	& \wahr		& \wahr		& \wahr		& \wahr		\\ \hline 
\wahr 	& \falsch 	& \falsch 	& \falsch 	& \falsch	& \falsch	& \falsch	& \falsch	\\ \hline 
\wahr 	& \falsch 	& \wahr 	& \falsch 	& \falsch	& \falsch	& \falsch	& \falsch	\\ \hline 
\wahr 	& \wahr 	& \falsch 	& \wahr 	& \wahr		& \wahr		& \wahr		& \wahr		\\ \hline 
\wahr 	& \wahr 	& \wahr 	& \wahr 	& \wahr		& \wahr		& \wahr		& \wahr		\\ \hline 
\end{tabular}
\end{enumerate}
\item\begin{enumerate}
\item Wahrheitstabelle\\
\begin{tabular}{|c|c|c|c|c|c|}
\hline
$A$ & $B$ & $(A\land(\neg B))$ & $(B\land(\neg A))$ & $(1 \lor 2)$ & $A \otimes B$\\ \hline
\falsch & \falsch 	& \falsch	& \falsch	& \falsch	& \falsch	\\ \hline 
\falsch & \wahr 	& \falsch 	& \wahr		& \wahr		& \wahr		\\ \hline 
\wahr 	& \falsch 	& \wahr 	& \falsch	& \wahr		& \wahr		\\ \hline 
\wahr 	& \wahr 	& \falsch 	& \falsch	& \falsch	& \falsch	\\ \hline 
\end{tabular}
\\\\
Wenn zwei Spalten die selben Werte haben(wahr,falsch) dann sind diese Ident $\equiv$
\item Wahrheitstabelle\\
\begin{tabular}{|c|c|c|c|c|c|}
\hline
$A$ & $B$ & $(A\Rightarrow B))$ & $(B\Rightarrow A))$ & $(1 \land 2)$ & $A \iff B$\\ \hline
\falsch & \falsch 	& \wahr		& \wahr		& \wahr		& \wahr		\\ \hline 
\falsch & \wahr 	& \wahr 	& \falsch	& \falsch	& \falsch	\\ \hline 
\wahr 	& \falsch 	& \falsch 	& \wahr		& \falsch	& \falsch	\\ \hline 
\wahr 	& \wahr 	& \wahr 	& \wahr		& \wahr		& \wahr		\\ \hline 
\end{tabular}
\end{enumerate}
\item $2^n$ bei zweistelligen Junktoren $2^4$
\end{enumerate}


\includepdf[offset=-10 -25,pages=1,pagecommand={\subsection{Übung 2}}]{Angabe/LGI_Uebung02.pdf}
\includepdf[pages=2-,offset=-10 -25,pagecommand={}]{Angabe/LGI_Uebung02.pdf}
\input{./LGI/LGI_Uebung02}



\includepdf[offset=-10 -25,pages=1,pagecommand={\subsection{Übung 3}}]{Angabe/LGI_Uebung03.pdf}
\includepdf[pages=2-,offset=-10 -25,pagecommand={}]{Angabe/LGI_Uebung03.pdf}
\input{./LGI/LGI_Uebung03}

\includepdf[offset=-10 -25,pages=1,pagecommand={\subsection{Übung 4}}]{Angabe/LGI_Uebung04.pdf}
\input{./LGI/LGI_Uebung04}

%EIR Übung%

\includepdf[offset=-10 -45,pages=1,pagecommand={\section{EIR Übung} \subsection{Übung 1}}]{Angabe/LGI_Uebung01.pdf}
\includepdf[pages=2-,offset=-10 -45,pagecommand={}]{Angabe/LGI_Uebung01.pdf}
\begin{enumerate}
\item
\begin{enumerate}
\item $\conj{\disj{((A\lor\conj{(B\land C)}}\land\nega{(\neg \impl{(A \Rightarrow B)}))})}$
\item $\impl{(\conj{(A\land B)}\Rightarrow\disj{(A\lor\nega{(\neg B)}}))}$
\item $\disj{\conj{(\impl{(A\Rightarrow B)}\land C)} \conj{\nega{(\neg A)}\land\nega{(\neg C)}}}$
\end{enumerate}
\item
\begin{enumerate}
\item Aussage A endet mit Implikation (Junktor) -> nicht wohlgeformt
\item eine Klammer zu viel
\item es fehlen umhüllende Klammern
\end{enumerate}
\item 
\begin{enumerate}
\item Wahrheitstabelle

\begin{tabular}{|c|c|c|c|c|}
\hline
$A$ & $B$ & $C$ & $B \land C$ & $(A \lor (B \land C)))$\\ \hline
\falsch & \falsch 	& \falsch 	& \falsch	& \falsch	\\ \hline 
\falsch & \falsch 	& \wahr 	& \falsch 	& \falsch	\\ \hline 
\falsch & \wahr 	& \falsch 	& \falsch 	& \falsch	\\ \hline 
\falsch & \wahr 	& \wahr 	& \wahr 	& \wahr		\\ \hline 
\wahr 	& \falsch 	& \falsch 	& \falsch 	& \wahr		\\ \hline 
\wahr 	& \falsch 	& \wahr 	& \falsch 	& \wahr		\\ \hline 
\wahr 	& \wahr 	& \falsch 	& \falsch 	& \wahr		\\ \hline 
\wahr 	& \wahr 	& \wahr 	& \wahr 	& \wahr		\\ \hline 
\end{tabular}

\item Wahrheitstabelle \\
\begin{tabular}{|c|c|c|c|c|}
\hline
$A$ & $B$ & $A \land B$ & $A \lor B$ & $((A \land B) \Rightarrow (A \lor B))$\\ \hline
\falsch & \falsch 	& \falsch 	& \falsch	& \wahr		\\ \hline 
\falsch & \falsch 	& \falsch 	& \wahr 	& \wahr		\\ \hline 
\wahr 	& \wahr 	& \falsch 	& \wahr 	& \wahr		\\ \hline 
\wahr	& \wahr 	& \wahr 	& \wahr 	& \wahr		\\ \hline 
\end{tabular}
\\\\
Tautologie.. 	Wenn alle in einer Spalte Richtig sind\\
Widerspruch.. 	Wenn alle in einer Spalte Falsch sind

\item Wahrheitstabelle \\
\begin{tabular}{|c|c|c|c|c|c|}
\hline
$A$ & $B$ & $C$ & $(A \Rightarrow B)$ & $((A \Rightarrow B) \land C)$ & $((A \Rightarrow B)\land C) \lor(\neg A))$\\ \hline
\falsch & \falsch 	& \falsch 	& \wahr		& \falsch	& \wahr		\\ \hline 
\falsch & \falsch 	& \wahr 	& \wahr 	& \wahr		& \wahr		\\ \hline 
\falsch & \wahr 	& \falsch 	& \wahr 	& \falsch	& \wahr		\\ \hline 
\falsch & \wahr 	& \wahr 	& \wahr 	& \wahr		& \wahr		\\ \hline 
\wahr 	& \falsch 	& \falsch 	& \falsch 	& \falsch	& \falsch	\\ \hline 
\wahr 	& \falsch 	& \wahr 	& \falsch 	& \falsch	& \falsch	\\ \hline 
\wahr 	& \wahr 	& \falsch 	& \wahr 	& \falsch	& \falsch	\\ \hline 
\wahr 	& \wahr 	& \wahr 	& \wahr 	& \wahr		& \wahr		\\ \hline 
\end{tabular}
\end{enumerate}

\newpage
\item
\begin{enumerate}
\item Wahrheitstabelle\\
\begin{tabular}{|c|c|c|c|c|c|c|}
\hline
$A$ & $B$ & $C$ & $(A \lor(\neg B \land A))$ & $(C \lor (B \lor A))$ & $(1 \land 2)$ & $A$\\ \hline
\falsch & \falsch 	& \falsch 	& \falsch	& \falsch	& \falsch	& \falsch	\\ \hline 
\falsch & \falsch 	& \wahr 	& \falsch 	& \wahr		& \falsch	& \falsch	\\ \hline 
\falsch & \wahr 	& \falsch 	& \falsch 	& \wahr		& \falsch	& \falsch	\\ \hline 
\falsch & \wahr 	& \wahr 	& \falsch 	& \wahr		& \falsch	& \falsch	\\ \hline 
\wahr 	& \falsch 	& \falsch 	& \wahr 	& \wahr		& \wahr		& \wahr		\\ \hline 
\wahr 	& \falsch 	& \wahr 	& \wahr 	& \wahr		& \wahr		& \wahr		\\ \hline 
\wahr 	& \wahr 	& \falsch 	& \wahr 	& \wahr		& \wahr		& \wahr		\\ \hline 
\wahr 	& \wahr 	& \wahr 	& \wahr 	& \wahr		& \wahr		& \wahr		\\ \hline 
\end{tabular}
\\\\
Wenn zwei Spalten die selben Werte haben(wahr,falsch) dann sind diese Ident $\equiv$
\item Wahrheitstabelle\\
\begin{tabular}{|c|c|c|c|c|c|c|c|}
\hline
$x$ & $y$ & $z$ & $\neg(x\land \neg y)$ & $(y\land (x \lor z))$ & $(1 \lor 2)$ & $\neg x\lor y$ & $x\Rightarrow y$\\ \hline
\falsch & \falsch 	& \falsch 	& \wahr		& \falsch	& \wahr		& \wahr		& \wahr		\\ \hline 
\falsch & \falsch 	& \wahr 	& \wahr 	& \falsch	& \wahr		& \wahr		& \wahr		\\ \hline 
\falsch & \wahr 	& \falsch 	& \wahr 	& \falsch	& \wahr		& \wahr		& \wahr		\\ \hline 
\falsch & \wahr 	& \wahr 	& \wahr 	& \wahr		& \wahr		& \wahr		& \wahr		\\ \hline 
\wahr 	& \falsch 	& \falsch 	& \falsch 	& \falsch	& \falsch	& \falsch	& \falsch	\\ \hline 
\wahr 	& \falsch 	& \wahr 	& \falsch 	& \falsch	& \falsch	& \falsch	& \falsch	\\ \hline 
\wahr 	& \wahr 	& \falsch 	& \wahr 	& \wahr		& \wahr		& \wahr		& \wahr		\\ \hline 
\wahr 	& \wahr 	& \wahr 	& \wahr 	& \wahr		& \wahr		& \wahr		& \wahr		\\ \hline 
\end{tabular}
\end{enumerate}
\item\begin{enumerate}
\item Wahrheitstabelle\\
\begin{tabular}{|c|c|c|c|c|c|}
\hline
$A$ & $B$ & $(A\land(\neg B))$ & $(B\land(\neg A))$ & $(1 \lor 2)$ & $A \otimes B$\\ \hline
\falsch & \falsch 	& \falsch	& \falsch	& \falsch	& \falsch	\\ \hline 
\falsch & \wahr 	& \falsch 	& \wahr		& \wahr		& \wahr		\\ \hline 
\wahr 	& \falsch 	& \wahr 	& \falsch	& \wahr		& \wahr		\\ \hline 
\wahr 	& \wahr 	& \falsch 	& \falsch	& \falsch	& \falsch	\\ \hline 
\end{tabular}
\\\\
Wenn zwei Spalten die selben Werte haben(wahr,falsch) dann sind diese Ident $\equiv$
\item Wahrheitstabelle\\
\begin{tabular}{|c|c|c|c|c|c|}
\hline
$A$ & $B$ & $(A\Rightarrow B))$ & $(B\Rightarrow A))$ & $(1 \land 2)$ & $A \iff B$\\ \hline
\falsch & \falsch 	& \wahr		& \wahr		& \wahr		& \wahr		\\ \hline 
\falsch & \wahr 	& \wahr 	& \falsch	& \falsch	& \falsch	\\ \hline 
\wahr 	& \falsch 	& \falsch 	& \wahr		& \falsch	& \falsch	\\ \hline 
\wahr 	& \wahr 	& \wahr 	& \wahr		& \wahr		& \wahr		\\ \hline 
\end{tabular}
\end{enumerate}
\item $2^n$ bei zweistelligen Junktoren $2^4$
\end{enumerate}

\includepdf[offset=-10 -25,pages=1,pagecommand={\subsection{Übung 2}}]{Angabe/LGI_Uebung02.pdf}
\includepdf[pages=2-,offset=-10 -25,pagecommand={}]{Angabe/LGI_Uebung02.pdf}
\input{./LGI/LGI_Uebung02}

\includepdf[offset=-10 -25,pages=1,pagecommand={\subsection{Übung 3}}]{Angabe/LGI_Uebung03.pdf}
\includepdf[pages=2-,offset=-10 -25,pagecommand={}]{Angabe/LGI_Uebung03.pdf}
\input{./LGI/LGI_Uebung03}



\end{document}





